\documentclass{beamer}

\usetheme{hsrm}
\usepackage{datetime}
\usepackage{hyperref}
\usepackage{booktabs}
\usepackage[spanish]{babel}

\title{Pensando en Greedy}
\subtitle{Programacion Competitiva}
\author{Santiago Hincapie Potes}
\institute{Universidad EAFIT}

\newdate{date}{22}{02}{2019}

\date{\displaydate{date}}

\begin{document}

\maketitle

\section*{Content}
\begin{frame}{El dia de hoy veremos}
  \tableofcontents[hideallsubsections]
\end{frame}

\section{Algoritmos greedy}
\begin{frame}{¿Que son?}
  \begin{itemize}
  \item Diremos que un algoritmo es greedy cuando en cada paso, elige la
    ``mejor'' solución local.
    \pause
  \item Dicha función de elección puede conducirnos o no a una solución
    óptima.
    \pause
  \item Cuando el algoritmo conduzca a una solución óptima diremos que
    el greedy ``funciona''.
    \pause
  \item Beneficio inmediato.
  \end{itemize}
\end{frame}

\begin{frame}{Problema de la moneda}
  \begin{itemize}
  \item El problema de cambio de monedas aborda la forma de encontrar el número mínimo
    de monedas (de ciertas denominaciones) tales que entre ellas suman una cierta
    cantidad. \pause
  \item Elegimos en cada paso la moneda de mayor denominacion que no supere el monto.
    \pause
  \item ¿Funciona esta idea?
    \pause
  \item Consideremos que tenemos monedas de $(25, 15, 1)$ y deseamos dar un cambio de $30$
    \pause
  \item El algoritmo encontraria la secuencia $\{25, 1, 1, 1, 1, 1\}$, sin embargo,
    la secuencia optima es $\{15, 15\}$
  \end{itemize}
\end{frame}

\begin{frame}{Problema de la Selección de Tareas}
  Juan tiene $n$ actividades que realizar y sabe cuándo empieza y
  cuándo termina cada una. Lamentablemente algunas se superponen y
  por lo tanto no puede realizarlas todas. El problema pide la máxima
  cantidad de actividades que Juan puede realizar sin que se le
  superpongan dos de ellas.
  \begin{itemize}
  \item Por ejemplo si tenemos tres tareas de rangos $(1,3)$ , $(2,9)$ y $(8,10)$
  \item $\dots$ la respuesta sería 2 tareas, la primera y la última.
  \end{itemize}
\end{frame}

\begin{frame}{Problema de la Selecci\'on de Tareas}
  \begin{itemize}
  \item ¿Hay alguna forma de decidir rápidamente qué tarea hacer primero? \pause 
  \item ¿elegir la tarea que dure menos tiempo? \pause
  \item ¿la tarea que empiece primero? \pause
  \item No funcionan. \pause
  \item \textbf{Clave:} Escoger la tare que te deje el mayor tiempo posible
    para realizar las próximas.
  \end{itemize}
\end{frame}

\begin{frame}{Problema de la Selección de Tareas}
  \begin{itemize}
  \item La forma correcta de ordenarlas es por horario de finalización\pause
  \item Siempre que podamos realizar la próxima tarea la realizamos,
    sino la ignoramos.\pause
  \item De esta forma, intuitivamente vamos realizando una a una las
    tareas con el objetivo de que nos sobre mayor tiempo para
    realizar las otras.\pause
  \item ¿Funciona esto?
  \end{itemize}
\end{frame}

\begin{frame}{¿Por qué es correcto este algoritmo?}
  \begin{itemize}
  \item Supongamos que el algoritmo no es óptimo\pause % Existe una selecion mejor
  \item Con la selección de tareas que nosotros realizamos vamos
    resolviendo los siguientes subproblemas: ¿Cuántas actividades
    podemos hacer desde que terminaron las primeras $i$ actividades?
  \item ¿Cuál es la próxima tarea a hacer?\pause
  \item Supongamos que en ese subproblema, no hay solución eligiendo
    como primer tarea la que finaliza primero dentro de las posibles.
  \item Borremos la primer tarea elegida, y pongamos la que finaliza
    primero de las posibles. Todas las otras claramente van a poder
    realizarse
  \item Por lo tanto hay una solución óptima que elije la primer tarea que
    finaliza. Contradicción.
  \item Luego, el algoritmo es óptimo
  \end{itemize}
\end{frame}

\begin{frame}{Teorema de Nico Alvarez}
  ``Todos los problemas Greedies salen igual. Hay que ordenar `las
  tareas' y después resolverlas en ese orden. Para ver en que orden se
  resuelven tenes que agarrar dos tareas y ver cual es la que
  greedymente se tiene que hacer primero''\pause
  Eso quiere decir que el código será simplemente:
  \begin{itemize}
  \item Hacer una función de comparación entre 2 tareas
  \item Ordenar el `arreglo de tareas'
  \item Hacer un \texttt{for}
  \end{itemize}
  La parte más difícil claramente es la función de comparacion
\end{frame}

\begin{frame}{The Hero}
Dado un héroe llamado Foronda con sus puntos de vida inicial y
dados los monstruos que Foronda tiene que matar, queremos saber
si puede matarlos a todos sin quedarse en ningún momento sin
energía.\\
Los monstruos se simbolizan con la vida $c_i$ que le cuesta al héroe
matar al $i$-ésimo monstruo. Además, cada monstruo cuida un cofre
que contiene una poción, la cual Foronda sólo puede beber luego de
matar al monstruo que la cuida y que le hace recuperar $r_i$ puntos de
vida al héroe. Foronda tiene vida máxima infinita.\\
\textbf{Constraints:}
$N \leq 10^5$ Monstruos, $1 \leq Z \leq 10^5$ vida inicial, $c_i
, r_i \leq 10^5$ naturales
\end{frame}

\begin{frame}{Solucion}
  \begin{itemize}
    \item Por el teorema anterior hay que buscar una forma de ordenar los
      monstruos para saber cuál matar primero.\pause
    \item Lo primero que hay que suponer es que podemos matar a todos
      y ver si llegamos a una contradicción.\pause
    \item Ahora... ¿En qué orden los matamos?\pause
    \item Empecemos matando a los monstruos buenos, los que te dan
      diferencia positiva de vida, es decir, los que la poción te da más
      vida que la que te saca el monstruo.\pause
    \item Supongamos que no los podemos matar al principio, no los
      vamos a poder matar teniendo menos vida y ademas después vamos 
      a tener más vida para los otros.
  \end{itemize}
\end{frame}

\begin{frame}{Solucion}
  \begin{itemize}
    \item Pero... ¿En qué orden matamos a los monstruos buenos?
    \item No parece muy complejo, como cada vez vamos a tener mayor
      vida si antes podíamos matar a un monstruo bueno, nunca
      vamos a dejar de poder matarlo, por lo que una estrategia “matar
      al que podamos” va a funcionar.
    \item Si en algún momento queda algún monstruo bueno y no podemos
      matar a ningun otro bueno, nunca podremos matarlo.
    \item Pensando un poquito más para simplificar el algoritmo, podemos
      matarlos en orden creciente de la vida $c_i$ que nos cuesta matarlos.
    \item Si en algún momento no podemos matar al monstruo bueno
      $i$-ésimo no podremos matar a ningún otro bueno ya que nunca
      podremos poseer más vida de la que tenemos.
  \end{itemize}
\end{frame}

\begin{frame}{Solucion}
  \begin{itemize}
    \item Nos quedan los malos.
    \item ¿Podemos matarlos en el mismo orden que a los buenos?
    \item Antes de programar esa idea, intentemos buscar un caso borde
      que nos destruya ese greedy...
    \item Si un monstruo malo cuesta mucho pero te recupera casi la
      misma cantidad quizá convenga matarlo antes, ¿no?
    \item ¿Otra idea?
    \item ¡Matemos al que te cueste menor diferencia de vida!
    \item ¡Matemos al que te saque mayor vida!
    \item Generemos un caso de prueba con 2 monstruos y
      analicemos como deberiamos resolverlos
  \end{itemize}
\end{frame}

\begin{frame}{Solucion}
  \begin{itemize}
  \item Matemos los monstruos malos en orden decreciente de lo que
    nos recuperan $r_i$.
  \item Pensemos un caso para probarlo (tablero)
  \item ¿Como podemos estar seguros de que esto funciona?
  \end{itemize}
\end{frame}

% otro ejemplo http://trainingcamp.elsantodel90.tk/actual/clases/Clase%202_%20Greedy.pdf

\section{Greedy is god}
\begin{frame}{Cuando podemos utilizar una estrategia greedy?}
  \begin{itemize}
  \item Cuando un problema exhibe la propiedad ‘optimal-substructure‘.
    Es decir que toda solución óptima a un problema puede ser
    construida considerando soluciones optimas de los
    subproblemas.
  \item Los problemas que exhiben dicha propiedad pueden ser resueltos
    de forma greedy o con programación dinamica.
  \item Existe también un conjunto de teoremas para demostrar que
    cuando un problema exhibe las propiedades de un matroide,
    siempre un algoritmo greedy nos llevará a una solución maximal
    que es óptima.
  \end{itemize}
\end{frame}

\begin{frame}{Matroide}
  Es una estructura  estructura que toma y generaliza el concepto de
  independencia lineal en los espacios vectoriales.\\\pause
  Formalmente un matroide $M$ es un par ordenado de elementos $(E,I)$ 
  donde $E$ es un conjunto finito e $I$ es un subconjunto del conjunto
  potencia de $E$ que cumplen las siguientes propiedades
  \begin{itemize}
  \item $\emptyset \in I$
  \item Si $A \in I$ y $B \subseteq A$ entonces $B \in I$
  \item Si $A, B \in I$ y $|B| < |A|$ entonces existe $e\in A - B$ 
    tal que $B\cup \{e\}\in I$ 
  \end{itemize}\pause
  Es la manera formal de probar que algo puede ser resuelto por un algoritmo Greedy 
\end{frame}

\begin{frame}{''Demostración'' de un algoritmo Greedy}
  \begin{itemize}
    \item Para tener garantizado un “Accepted”, los Greedys hay que
      demostrarlos.
    \item Es raro que alguien utilice matroides en programacion competitiva.
    \item Usualmente lo que se hace es trabajar con reduccion al absurdo. 
    \item O.... se puede probar con casos de prueba inteligentes.
    \item Casos de pruebas inteligentes son unos pocos casos chicos o
      bien pensados, donde poder analizar que ideas sirven.
    \item Este enfoque es riesgoso, uno tiene que estar preparado para
      dejar un problema si no te da Accepted.
    \item Sin embargo suele ser muy efectivo.
  \end{itemize}
\end{frame}

\begin{frame}{Otros usos de los casos de pruebas}
  \begin{itemize}
    \item Podemos usarlos antes de codear para garantizar que no
      estemos programando cualquier cosa.
    \item Podemos usarlos para comprobar y buscar algoritmos.
    \item Y podemos usarlos para entender los problemas 
      (poco comun en greedy, pero muy importante).
  \end{itemize}
\end{frame}

\begin{frame}{Woodworms}
  \begin{itemize}
    \item La mejor forma de aprender a resolver Greedys es realizando
      problemas
    \item Para demostrar que andan, se procede por el absurdo. Se supone
      que el greedy no es óptimo y se llega a una contradicción
    \item Otra forma es probar casos extremos, y confiar en que funciona
    \item Como los greedys son muy simples de codear, lo mandamos y
      probamos. Si pasan estamos melos, sino lo volvemos a pensar.
    \item Para encontrar un greedy en un ejercicio es fundamental probar
      como resolverías el ejercicio para casos extremos simples.
  \end{itemize}
\end{frame}

\section{Proxima sesion}
\begin{frame}{Contest}
  \centering
  \url{https://vjudge.net/contest/284280}
\end{frame}

\begin{frame}{Proxima semana}
  \centering 
  \Large
  Binary search
\end{frame}
\end{document}
