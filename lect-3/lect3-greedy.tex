\documentclass{beamer}

\usetheme{hsrm}
\usepackage{datetime}
\usepackage{hyperref}
\usepackage{booktabs}
\usepackage[spanish]{babel}

\title{Pensando en Greedy}
\subtitle{Programacion Competitiva}
\author{Santiago Hincapie Potes}
\institute{Universidad EAFIT}

\newdate{date}{22}{02}{2019}

\date{\displaydate{date}}

\begin{document}

\maketitle

\section*{Content}
\begin{frame}{El dia de hoy veremos}
  \tableofcontents[hideallsubsections]
\end{frame}

\section{Algoritmos greedy}
\begin{frame}{¿Que son?}
  \begin{itemize}
  \item Diremos que un algoritmo es greedy cuando en cada paso, elige la
    ``mejor'' solución local.
    \pause
  \item Dicha función de elección puede conducirnos o no a una solución
    óptima.
    \pause
  \item Cuando el algoritmo conduzca a una solución óptima diremos que
    el greedy ``funciona''.
    \pause
  \item Beneficio inmediato.
  \end{itemize}
\end{frame}

\begin{frame}{Problema de la moneda}
  \begin{itemize}
  \item El problema de cambio de monedas aborda la forma de encontrar el número mínimo
    de monedas (de ciertas denominaciones) tales que entre ellas suman una cierta
    cantidad. \pause
  \item Elegimos en cada paso la moneda de mayor denominacion que no supere el monto.
    \pause
  \item ¿Funciona esta idea?
    \pause
  \item Consideremos que tenemos monedas de $(25, 15, 1)$ y deseamos dar un cambio de $30$
    \pause
  \item El algoritmo encontraria la secuencia $\{25, 1, 1, 1, 1, 1\}$, sin embargo,
    la secuencia optima es $\{15, 15\}$
  \end{itemize}
\end{frame}

\begin{frame}{Problema de la Selección de Tareas}
  Juan tiene $n$ actividades que realizar y sabe cuándo empieza y
  cuándo termina cada una. Lamentablemente algunas se superponen y
  por lo tanto no puede realizarlas todas. El problema pide la máxima
  cantidad de actividades que Juan puede realizar sin que se le
  superpongan dos de ellas.
  \begin{itemize}
  \item Por ejemplo si tenemos tres tareas de rangos $(1,3)$ , $(2,9)$ y $(8,10)$
  \item $\dots$ la respuesta sería 2 tareas, la primera y la última.
  \end{itemize}
\end{frame}

\begin{frame}{Problema de la Selecci\'on de Tareas}
  \begin{itemize}
  \item ¿Hay alguna forma de decidir rápidamente qué tarea hacer primero? \pause 
  \item ¿elegir la tarea que dure menos tiempo? \pause
  \item ¿la tarea que empiece primero? \pause
  \item No funcionan. \pause
  \item \textbf{Clave:} Escoger la tare que te deje el mayor tiempo posible
    para realizar las próximas.
  \end{itemize}
\end{frame}

\begin{frame}{Problema de la Selección de Tareas}
  \begin{itemize}
  \item La forma correcta de ordenarlas es por horario de finalización
  \item Siempre que podamos realizar la próxima tarea la realizamos,
    sino la ignoramos.
  \item De esta forma, intuitivamente vamos realizando una a una las
    tareas con el objetivo de que nos sobre mayor tiempo para
    realizar las otras.
  \item ¿Funciona esto?
  \end{itemize}
\end{frame}

\begin{frame}{¿Por qué es correcto este algoritmo?}
  \begin{itemize}
  \item Supongamos que el algoritmo no es óptimo
  \item Con la selección de tareas que nosotros realizamos vamos
    resolviendo los siguientes subproblemas: ¿Cuántas actividades
    podemos hacer desde que terminaron las primeras i actividades?
    ¿Cuál es la próxima tarea a hacer?
  \item Supongamos que en ese subproblema, no hay solución eligiendo
    como primer tarea la que finaliza primero dentro de las posibles.
  \item Borremos la primer tarea elegida, y pongamos la que finaliza
    primero de las posibles. Todas las otras claramente van a poder
    realizarse
  \item Por lo tanto hay una solución óptima que elije la primer tarea que
    finaliza. Contradicción.
  \item Luego, el algoritmo es óptimo
  \end{itemize}
\end{frame}

\begin{frame}{Teorema de Nico Alvarez}
  ``Todos los problemas Greedies salen igual. Hay que ordenar `las
  tareas' y después resolverlas en ese orden. Para ver en que orden se
  resuelven tenes que agarrar dos tareas y ver cual es la que
  greedymente se tiene que hacer primero''\pause
  Eso quiere decir que el código será simplemente:
  \begin{itemize}
  \item Hacer una función de comparación entre 2 tareas
  \item Ordenar el `arreglo de tareas'
  \item Hacer un \texttt{for}
  \end{itemize}
  La parte más difícil claramente es la función de comparacion
\end{frame}

% otro ejemplo http://trainingcamp.elsantodel90.tk/actual/clases/Clase%202_%20Greedy.pdf

\section{Greedy is god}
% http://trainingcamp.elsantodel90.tk/anteriores/2012/clases/03-Greedy.pdf
%\subsection{¿Cuando podemos utilizar una estrategia greedy?}
% Metroid https://stackoverflow.com/questions/13557979/why-does-the-greedy-coin-change-algorithm-not-work-for-some-coin-sets 
%\subsection{¿Como podemos demostrar que una estategia greedy funcionará?}



\section{Proxima sesion}
\begin{frame}{Contest}
  TODO
\end{frame}

\begin{frame}{Next week}
  \centering 
  \Large
  ¿Como son los problemas de una maraton de programacion?
\end{frame}
\end{document}
